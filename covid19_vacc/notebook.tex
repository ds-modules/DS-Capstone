
% Default to the notebook output style

    


% Inherit from the specified cell style.




    
\documentclass[11pt]{article}

    
    
    \usepackage[T1]{fontenc}
    % Nicer default font (+ math font) than Computer Modern for most use cases
    \usepackage{mathpazo}

    % Basic figure setup, for now with no caption control since it's done
    % automatically by Pandoc (which extracts ![](path) syntax from Markdown).
    \usepackage{graphicx}
    % We will generate all images so they have a width \maxwidth. This means
    % that they will get their normal width if they fit onto the page, but
    % are scaled down if they would overflow the margins.
    \makeatletter
    \def\maxwidth{\ifdim\Gin@nat@width>\linewidth\linewidth
    \else\Gin@nat@width\fi}
    \makeatother
    \let\Oldincludegraphics\includegraphics
    % Set max figure width to be 80% of text width, for now hardcoded.
    \renewcommand{\includegraphics}[1]{\Oldincludegraphics[width=.8\maxwidth]{#1}}
    % Ensure that by default, figures have no caption (until we provide a
    % proper Figure object with a Caption API and a way to capture that
    % in the conversion process - todo).
    \usepackage{caption}
    \DeclareCaptionLabelFormat{nolabel}{}
    \captionsetup{labelformat=nolabel}

    \usepackage{adjustbox} % Used to constrain images to a maximum size 
    \usepackage{xcolor} % Allow colors to be defined
    \usepackage{enumerate} % Needed for markdown enumerations to work
    \usepackage{geometry} % Used to adjust the document margins
    \usepackage{amsmath} % Equations
    \usepackage{amssymb} % Equations
    \usepackage{textcomp} % defines textquotesingle
    % Hack from http://tex.stackexchange.com/a/47451/13684:
    \AtBeginDocument{%
        \def\PYZsq{\textquotesingle}% Upright quotes in Pygmentized code
    }
    \usepackage{upquote} % Upright quotes for verbatim code
    \usepackage{eurosym} % defines \euro
    \usepackage[mathletters]{ucs} % Extended unicode (utf-8) support
    \usepackage[utf8x]{inputenc} % Allow utf-8 characters in the tex document
    \usepackage{fancyvrb} % verbatim replacement that allows latex
    \usepackage{grffile} % extends the file name processing of package graphics 
                         % to support a larger range 
    % The hyperref package gives us a pdf with properly built
    % internal navigation ('pdf bookmarks' for the table of contents,
    % internal cross-reference links, web links for URLs, etc.)
    \usepackage{hyperref}
    \usepackage{longtable} % longtable support required by pandoc >1.10
    \usepackage{booktabs}  % table support for pandoc > 1.12.2
    \usepackage[inline]{enumitem} % IRkernel/repr support (it uses the enumerate* environment)
    \usepackage[normalem]{ulem} % ulem is needed to support strikethroughs (\sout)
                                % normalem makes italics be italics, not underlines
    

    
    
    % Colors for the hyperref package
    \definecolor{urlcolor}{rgb}{0,.145,.698}
    \definecolor{linkcolor}{rgb}{.71,0.21,0.01}
    \definecolor{citecolor}{rgb}{.12,.54,.11}

    % ANSI colors
    \definecolor{ansi-black}{HTML}{3E424D}
    \definecolor{ansi-black-intense}{HTML}{282C36}
    \definecolor{ansi-red}{HTML}{E75C58}
    \definecolor{ansi-red-intense}{HTML}{B22B31}
    \definecolor{ansi-green}{HTML}{00A250}
    \definecolor{ansi-green-intense}{HTML}{007427}
    \definecolor{ansi-yellow}{HTML}{DDB62B}
    \definecolor{ansi-yellow-intense}{HTML}{B27D12}
    \definecolor{ansi-blue}{HTML}{208FFB}
    \definecolor{ansi-blue-intense}{HTML}{0065CA}
    \definecolor{ansi-magenta}{HTML}{D160C4}
    \definecolor{ansi-magenta-intense}{HTML}{A03196}
    \definecolor{ansi-cyan}{HTML}{60C6C8}
    \definecolor{ansi-cyan-intense}{HTML}{258F8F}
    \definecolor{ansi-white}{HTML}{C5C1B4}
    \definecolor{ansi-white-intense}{HTML}{A1A6B2}

    % commands and environments needed by pandoc snippets
    % extracted from the output of `pandoc -s`
    \providecommand{\tightlist}{%
      \setlength{\itemsep}{0pt}\setlength{\parskip}{0pt}}
    \DefineVerbatimEnvironment{Highlighting}{Verbatim}{commandchars=\\\{\}}
    % Add ',fontsize=\small' for more characters per line
    \newenvironment{Shaded}{}{}
    \newcommand{\KeywordTok}[1]{\textcolor[rgb]{0.00,0.44,0.13}{\textbf{{#1}}}}
    \newcommand{\DataTypeTok}[1]{\textcolor[rgb]{0.56,0.13,0.00}{{#1}}}
    \newcommand{\DecValTok}[1]{\textcolor[rgb]{0.25,0.63,0.44}{{#1}}}
    \newcommand{\BaseNTok}[1]{\textcolor[rgb]{0.25,0.63,0.44}{{#1}}}
    \newcommand{\FloatTok}[1]{\textcolor[rgb]{0.25,0.63,0.44}{{#1}}}
    \newcommand{\CharTok}[1]{\textcolor[rgb]{0.25,0.44,0.63}{{#1}}}
    \newcommand{\StringTok}[1]{\textcolor[rgb]{0.25,0.44,0.63}{{#1}}}
    \newcommand{\CommentTok}[1]{\textcolor[rgb]{0.38,0.63,0.69}{\textit{{#1}}}}
    \newcommand{\OtherTok}[1]{\textcolor[rgb]{0.00,0.44,0.13}{{#1}}}
    \newcommand{\AlertTok}[1]{\textcolor[rgb]{1.00,0.00,0.00}{\textbf{{#1}}}}
    \newcommand{\FunctionTok}[1]{\textcolor[rgb]{0.02,0.16,0.49}{{#1}}}
    \newcommand{\RegionMarkerTok}[1]{{#1}}
    \newcommand{\ErrorTok}[1]{\textcolor[rgb]{1.00,0.00,0.00}{\textbf{{#1}}}}
    \newcommand{\NormalTok}[1]{{#1}}
    
    % Additional commands for more recent versions of Pandoc
    \newcommand{\ConstantTok}[1]{\textcolor[rgb]{0.53,0.00,0.00}{{#1}}}
    \newcommand{\SpecialCharTok}[1]{\textcolor[rgb]{0.25,0.44,0.63}{{#1}}}
    \newcommand{\VerbatimStringTok}[1]{\textcolor[rgb]{0.25,0.44,0.63}{{#1}}}
    \newcommand{\SpecialStringTok}[1]{\textcolor[rgb]{0.73,0.40,0.53}{{#1}}}
    \newcommand{\ImportTok}[1]{{#1}}
    \newcommand{\DocumentationTok}[1]{\textcolor[rgb]{0.73,0.13,0.13}{\textit{{#1}}}}
    \newcommand{\AnnotationTok}[1]{\textcolor[rgb]{0.38,0.63,0.69}{\textbf{\textit{{#1}}}}}
    \newcommand{\CommentVarTok}[1]{\textcolor[rgb]{0.38,0.63,0.69}{\textbf{\textit{{#1}}}}}
    \newcommand{\VariableTok}[1]{\textcolor[rgb]{0.10,0.09,0.49}{{#1}}}
    \newcommand{\ControlFlowTok}[1]{\textcolor[rgb]{0.00,0.44,0.13}{\textbf{{#1}}}}
    \newcommand{\OperatorTok}[1]{\textcolor[rgb]{0.40,0.40,0.40}{{#1}}}
    \newcommand{\BuiltInTok}[1]{{#1}}
    \newcommand{\ExtensionTok}[1]{{#1}}
    \newcommand{\PreprocessorTok}[1]{\textcolor[rgb]{0.74,0.48,0.00}{{#1}}}
    \newcommand{\AttributeTok}[1]{\textcolor[rgb]{0.49,0.56,0.16}{{#1}}}
    \newcommand{\InformationTok}[1]{\textcolor[rgb]{0.38,0.63,0.69}{\textbf{\textit{{#1}}}}}
    \newcommand{\WarningTok}[1]{\textcolor[rgb]{0.38,0.63,0.69}{\textbf{\textit{{#1}}}}}
    
    
    % Define a nice break command that doesn't care if a line doesn't already
    % exist.
    \def\br{\hspace*{\fill} \\* }
    % Math Jax compatability definitions
    \def\gt{>}
    \def\lt{<}
    % Document parameters
    \title{covid19\_vacc\_prog}
    
    
    

    % Pygments definitions
    
\makeatletter
\def\PY@reset{\let\PY@it=\relax \let\PY@bf=\relax%
    \let\PY@ul=\relax \let\PY@tc=\relax%
    \let\PY@bc=\relax \let\PY@ff=\relax}
\def\PY@tok#1{\csname PY@tok@#1\endcsname}
\def\PY@toks#1+{\ifx\relax#1\empty\else%
    \PY@tok{#1}\expandafter\PY@toks\fi}
\def\PY@do#1{\PY@bc{\PY@tc{\PY@ul{%
    \PY@it{\PY@bf{\PY@ff{#1}}}}}}}
\def\PY#1#2{\PY@reset\PY@toks#1+\relax+\PY@do{#2}}

\expandafter\def\csname PY@tok@w\endcsname{\def\PY@tc##1{\textcolor[rgb]{0.73,0.73,0.73}{##1}}}
\expandafter\def\csname PY@tok@c\endcsname{\let\PY@it=\textit\def\PY@tc##1{\textcolor[rgb]{0.25,0.50,0.50}{##1}}}
\expandafter\def\csname PY@tok@cp\endcsname{\def\PY@tc##1{\textcolor[rgb]{0.74,0.48,0.00}{##1}}}
\expandafter\def\csname PY@tok@k\endcsname{\let\PY@bf=\textbf\def\PY@tc##1{\textcolor[rgb]{0.00,0.50,0.00}{##1}}}
\expandafter\def\csname PY@tok@kp\endcsname{\def\PY@tc##1{\textcolor[rgb]{0.00,0.50,0.00}{##1}}}
\expandafter\def\csname PY@tok@kt\endcsname{\def\PY@tc##1{\textcolor[rgb]{0.69,0.00,0.25}{##1}}}
\expandafter\def\csname PY@tok@o\endcsname{\def\PY@tc##1{\textcolor[rgb]{0.40,0.40,0.40}{##1}}}
\expandafter\def\csname PY@tok@ow\endcsname{\let\PY@bf=\textbf\def\PY@tc##1{\textcolor[rgb]{0.67,0.13,1.00}{##1}}}
\expandafter\def\csname PY@tok@nb\endcsname{\def\PY@tc##1{\textcolor[rgb]{0.00,0.50,0.00}{##1}}}
\expandafter\def\csname PY@tok@nf\endcsname{\def\PY@tc##1{\textcolor[rgb]{0.00,0.00,1.00}{##1}}}
\expandafter\def\csname PY@tok@nc\endcsname{\let\PY@bf=\textbf\def\PY@tc##1{\textcolor[rgb]{0.00,0.00,1.00}{##1}}}
\expandafter\def\csname PY@tok@nn\endcsname{\let\PY@bf=\textbf\def\PY@tc##1{\textcolor[rgb]{0.00,0.00,1.00}{##1}}}
\expandafter\def\csname PY@tok@ne\endcsname{\let\PY@bf=\textbf\def\PY@tc##1{\textcolor[rgb]{0.82,0.25,0.23}{##1}}}
\expandafter\def\csname PY@tok@nv\endcsname{\def\PY@tc##1{\textcolor[rgb]{0.10,0.09,0.49}{##1}}}
\expandafter\def\csname PY@tok@no\endcsname{\def\PY@tc##1{\textcolor[rgb]{0.53,0.00,0.00}{##1}}}
\expandafter\def\csname PY@tok@nl\endcsname{\def\PY@tc##1{\textcolor[rgb]{0.63,0.63,0.00}{##1}}}
\expandafter\def\csname PY@tok@ni\endcsname{\let\PY@bf=\textbf\def\PY@tc##1{\textcolor[rgb]{0.60,0.60,0.60}{##1}}}
\expandafter\def\csname PY@tok@na\endcsname{\def\PY@tc##1{\textcolor[rgb]{0.49,0.56,0.16}{##1}}}
\expandafter\def\csname PY@tok@nt\endcsname{\let\PY@bf=\textbf\def\PY@tc##1{\textcolor[rgb]{0.00,0.50,0.00}{##1}}}
\expandafter\def\csname PY@tok@nd\endcsname{\def\PY@tc##1{\textcolor[rgb]{0.67,0.13,1.00}{##1}}}
\expandafter\def\csname PY@tok@s\endcsname{\def\PY@tc##1{\textcolor[rgb]{0.73,0.13,0.13}{##1}}}
\expandafter\def\csname PY@tok@sd\endcsname{\let\PY@it=\textit\def\PY@tc##1{\textcolor[rgb]{0.73,0.13,0.13}{##1}}}
\expandafter\def\csname PY@tok@si\endcsname{\let\PY@bf=\textbf\def\PY@tc##1{\textcolor[rgb]{0.73,0.40,0.53}{##1}}}
\expandafter\def\csname PY@tok@se\endcsname{\let\PY@bf=\textbf\def\PY@tc##1{\textcolor[rgb]{0.73,0.40,0.13}{##1}}}
\expandafter\def\csname PY@tok@sr\endcsname{\def\PY@tc##1{\textcolor[rgb]{0.73,0.40,0.53}{##1}}}
\expandafter\def\csname PY@tok@ss\endcsname{\def\PY@tc##1{\textcolor[rgb]{0.10,0.09,0.49}{##1}}}
\expandafter\def\csname PY@tok@sx\endcsname{\def\PY@tc##1{\textcolor[rgb]{0.00,0.50,0.00}{##1}}}
\expandafter\def\csname PY@tok@m\endcsname{\def\PY@tc##1{\textcolor[rgb]{0.40,0.40,0.40}{##1}}}
\expandafter\def\csname PY@tok@gh\endcsname{\let\PY@bf=\textbf\def\PY@tc##1{\textcolor[rgb]{0.00,0.00,0.50}{##1}}}
\expandafter\def\csname PY@tok@gu\endcsname{\let\PY@bf=\textbf\def\PY@tc##1{\textcolor[rgb]{0.50,0.00,0.50}{##1}}}
\expandafter\def\csname PY@tok@gd\endcsname{\def\PY@tc##1{\textcolor[rgb]{0.63,0.00,0.00}{##1}}}
\expandafter\def\csname PY@tok@gi\endcsname{\def\PY@tc##1{\textcolor[rgb]{0.00,0.63,0.00}{##1}}}
\expandafter\def\csname PY@tok@gr\endcsname{\def\PY@tc##1{\textcolor[rgb]{1.00,0.00,0.00}{##1}}}
\expandafter\def\csname PY@tok@ge\endcsname{\let\PY@it=\textit}
\expandafter\def\csname PY@tok@gs\endcsname{\let\PY@bf=\textbf}
\expandafter\def\csname PY@tok@gp\endcsname{\let\PY@bf=\textbf\def\PY@tc##1{\textcolor[rgb]{0.00,0.00,0.50}{##1}}}
\expandafter\def\csname PY@tok@go\endcsname{\def\PY@tc##1{\textcolor[rgb]{0.53,0.53,0.53}{##1}}}
\expandafter\def\csname PY@tok@gt\endcsname{\def\PY@tc##1{\textcolor[rgb]{0.00,0.27,0.87}{##1}}}
\expandafter\def\csname PY@tok@err\endcsname{\def\PY@bc##1{\setlength{\fboxsep}{0pt}\fcolorbox[rgb]{1.00,0.00,0.00}{1,1,1}{\strut ##1}}}
\expandafter\def\csname PY@tok@kc\endcsname{\let\PY@bf=\textbf\def\PY@tc##1{\textcolor[rgb]{0.00,0.50,0.00}{##1}}}
\expandafter\def\csname PY@tok@kd\endcsname{\let\PY@bf=\textbf\def\PY@tc##1{\textcolor[rgb]{0.00,0.50,0.00}{##1}}}
\expandafter\def\csname PY@tok@kn\endcsname{\let\PY@bf=\textbf\def\PY@tc##1{\textcolor[rgb]{0.00,0.50,0.00}{##1}}}
\expandafter\def\csname PY@tok@kr\endcsname{\let\PY@bf=\textbf\def\PY@tc##1{\textcolor[rgb]{0.00,0.50,0.00}{##1}}}
\expandafter\def\csname PY@tok@bp\endcsname{\def\PY@tc##1{\textcolor[rgb]{0.00,0.50,0.00}{##1}}}
\expandafter\def\csname PY@tok@fm\endcsname{\def\PY@tc##1{\textcolor[rgb]{0.00,0.00,1.00}{##1}}}
\expandafter\def\csname PY@tok@vc\endcsname{\def\PY@tc##1{\textcolor[rgb]{0.10,0.09,0.49}{##1}}}
\expandafter\def\csname PY@tok@vg\endcsname{\def\PY@tc##1{\textcolor[rgb]{0.10,0.09,0.49}{##1}}}
\expandafter\def\csname PY@tok@vi\endcsname{\def\PY@tc##1{\textcolor[rgb]{0.10,0.09,0.49}{##1}}}
\expandafter\def\csname PY@tok@vm\endcsname{\def\PY@tc##1{\textcolor[rgb]{0.10,0.09,0.49}{##1}}}
\expandafter\def\csname PY@tok@sa\endcsname{\def\PY@tc##1{\textcolor[rgb]{0.73,0.13,0.13}{##1}}}
\expandafter\def\csname PY@tok@sb\endcsname{\def\PY@tc##1{\textcolor[rgb]{0.73,0.13,0.13}{##1}}}
\expandafter\def\csname PY@tok@sc\endcsname{\def\PY@tc##1{\textcolor[rgb]{0.73,0.13,0.13}{##1}}}
\expandafter\def\csname PY@tok@dl\endcsname{\def\PY@tc##1{\textcolor[rgb]{0.73,0.13,0.13}{##1}}}
\expandafter\def\csname PY@tok@s2\endcsname{\def\PY@tc##1{\textcolor[rgb]{0.73,0.13,0.13}{##1}}}
\expandafter\def\csname PY@tok@sh\endcsname{\def\PY@tc##1{\textcolor[rgb]{0.73,0.13,0.13}{##1}}}
\expandafter\def\csname PY@tok@s1\endcsname{\def\PY@tc##1{\textcolor[rgb]{0.73,0.13,0.13}{##1}}}
\expandafter\def\csname PY@tok@mb\endcsname{\def\PY@tc##1{\textcolor[rgb]{0.40,0.40,0.40}{##1}}}
\expandafter\def\csname PY@tok@mf\endcsname{\def\PY@tc##1{\textcolor[rgb]{0.40,0.40,0.40}{##1}}}
\expandafter\def\csname PY@tok@mh\endcsname{\def\PY@tc##1{\textcolor[rgb]{0.40,0.40,0.40}{##1}}}
\expandafter\def\csname PY@tok@mi\endcsname{\def\PY@tc##1{\textcolor[rgb]{0.40,0.40,0.40}{##1}}}
\expandafter\def\csname PY@tok@il\endcsname{\def\PY@tc##1{\textcolor[rgb]{0.40,0.40,0.40}{##1}}}
\expandafter\def\csname PY@tok@mo\endcsname{\def\PY@tc##1{\textcolor[rgb]{0.40,0.40,0.40}{##1}}}
\expandafter\def\csname PY@tok@ch\endcsname{\let\PY@it=\textit\def\PY@tc##1{\textcolor[rgb]{0.25,0.50,0.50}{##1}}}
\expandafter\def\csname PY@tok@cm\endcsname{\let\PY@it=\textit\def\PY@tc##1{\textcolor[rgb]{0.25,0.50,0.50}{##1}}}
\expandafter\def\csname PY@tok@cpf\endcsname{\let\PY@it=\textit\def\PY@tc##1{\textcolor[rgb]{0.25,0.50,0.50}{##1}}}
\expandafter\def\csname PY@tok@c1\endcsname{\let\PY@it=\textit\def\PY@tc##1{\textcolor[rgb]{0.25,0.50,0.50}{##1}}}
\expandafter\def\csname PY@tok@cs\endcsname{\let\PY@it=\textit\def\PY@tc##1{\textcolor[rgb]{0.25,0.50,0.50}{##1}}}

\def\PYZbs{\char`\\}
\def\PYZus{\char`\_}
\def\PYZob{\char`\{}
\def\PYZcb{\char`\}}
\def\PYZca{\char`\^}
\def\PYZam{\char`\&}
\def\PYZlt{\char`\<}
\def\PYZgt{\char`\>}
\def\PYZsh{\char`\#}
\def\PYZpc{\char`\%}
\def\PYZdl{\char`\$}
\def\PYZhy{\char`\-}
\def\PYZsq{\char`\'}
\def\PYZdq{\char`\"}
\def\PYZti{\char`\~}
% for compatibility with earlier versions
\def\PYZat{@}
\def\PYZlb{[}
\def\PYZrb{]}
\makeatother


    % Exact colors from NB
    \definecolor{incolor}{rgb}{0.0, 0.0, 0.5}
    \definecolor{outcolor}{rgb}{0.545, 0.0, 0.0}



    
    % Prevent overflowing lines due to hard-to-break entities
    \sloppy 
    % Setup hyperref package
    \hypersetup{
      breaklinks=true,  % so long urls are correctly broken across lines
      colorlinks=true,
      urlcolor=urlcolor,
      linkcolor=linkcolor,
      citecolor=citecolor,
      }
    % Slightly bigger margins than the latex defaults
    
    \geometry{verbose,tmargin=1in,bmargin=1in,lmargin=1in,rmargin=1in}
    
    

    \begin{document}
    
    
    \maketitle
    
    

    
    \section{World Vaccination Progress
EDA}\label{world-vaccination-progress-eda}

Okay, we have a vaccine; that's exciting news! But how are we doing? Is
the vaccination progressing quick enough? This lecture, we will explore
a dataset downloaded from \texttt{kaggle}, you can follow
\href{https://www.kaggle.com/gpreda/covid-world-vaccination-progress}{this
link} to learn more about this dataset. The dataset is collected by the
organization \href{https://ourworldindata.org/}{Our World in Data} that
publish papers about solutions to world issues.

    \begin{Verbatim}[commandchars=\\\{\}]
{\color{incolor}In [{\color{incolor}1}]:} \PY{c+c1}{\PYZsh{} list of imports}
        \PY{k+kn}{import} \PY{n+nn}{scipy} \PY{k}{as} \PY{n+nn}{sp}
        \PY{k+kn}{import} \PY{n+nn}{numpy} \PY{k}{as} \PY{n+nn}{np}
        \PY{k+kn}{import} \PY{n+nn}{warnings}
        \PY{n}{warnings}\PY{o}{.}\PY{n}{simplefilter}\PY{p}{(}\PY{n}{action}\PY{o}{=}\PY{l+s+s1}{\PYZsq{}}\PY{l+s+s1}{ignore}\PY{l+s+s1}{\PYZsq{}}\PY{p}{,} \PY{n}{category}\PY{o}{=}\PY{n+ne}{FutureWarning}\PY{p}{)}
        
        \PY{k+kn}{import} \PY{n+nn}{pandas} \PY{k}{as} \PY{n+nn}{pd}
        \PY{k+kn}{import} \PY{n+nn}{datascience}
        \PY{k+kn}{import} \PY{n+nn}{matplotlib}\PY{n+nn}{.}\PY{n+nn}{pyplot} \PY{k}{as} \PY{n+nn}{plt}
        \PY{o}{\PYZpc{}}\PY{k}{matplotlib} inline
        \PY{k+kn}{from} \PY{n+nn}{datascience} \PY{k}{import} \PY{n}{Table}
        \PY{k+kn}{import} \PY{n+nn}{seaborn} \PY{k}{as} \PY{n+nn}{sns}
\end{Verbatim}


    We very briefly walked through how to interact with an API (it's very
simple!) If you are interested, you can look at more in the official
\texttt{github} page: https://github.com/Kaggle/kaggle-api

    \begin{Verbatim}[commandchars=\\\{\}]
{\color{incolor}In [{\color{incolor}2}]:} \PY{c+c1}{\PYZsh{} kaggle API}
        \PY{o}{!} kaggle datasets list \PYZhy{}s \PY{l+s+s2}{\PYZdq{}covid vaccination\PYZdq{}} \PYZhy{}\PYZhy{}sort\PYZhy{}by \PY{l+s+s1}{\PYZsq{}hottest\PYZsq{}} \PY{p}{|} head \PYZhy{}10
\end{Verbatim}


    \begin{Verbatim}[commandchars=\\\{\}]
ref                                                title                                              size  lastUpdated          downloadCount  voteCount  usabilityRating  

-------------------------------------------------  ------------------------------------------------  -----  -------------------  -------------  ---------  ---------------  

gpreda/covid-world-vaccination-progress            COVID-19 World Vaccination Progress               101KB  2021-03-01 08:15:06          19482        888  1.0              

fedesoriano/coronavirus-covid19-vaccinations-data  COVID-19 World Vaccination Progress Data            3MB  2021-01-27 06:37:50            186          7  1.0              

keplaxo/twitter-vaccination-dataset                Twitter Vaccination Dataset                       305MB  2020-04-15 17:33:37            346         13  0.9411765        

padmajabuggaveeti/covid-vaccination-dataset-2021   COVID VACCINATION DATASET - 2021                   29KB  2021-01-27 11:04:44             75          5  0.5882353        

mpwolke/cusersmarildownloadsvaccinationcsv         Covid-19 Vaccination Doses Administered            681B  2020-12-28 21:06:44              7          2  1.0              

teesoong/covid-vaccination-forecast                Covid Vaccination forecast                          2KB  2021-02-27 14:33:03             10          2  0.47058824       

alechelyar/facebook-antivaccination-dataset        Facebook Anti-Vaccination Dataset                  53MB  2019-04-02 17:24:22            206          8  0.3529412        

rtatman/animal-bites                               Animal Bites                                       95KB  2017-09-15 17:21:38           3496         60  0.85294116       


    \end{Verbatim}

    \begin{Verbatim}[commandchars=\\\{\}]
{\color{incolor}In [{\color{incolor}4}]:} \PY{c+c1}{\PYZsh{} find the files in dataset}
        \PY{o}{!} kaggle datasets files gpreda/covid\PYZhy{}world\PYZhy{}vaccination\PYZhy{}progress \PYZhy{}v
\end{Verbatim}


    \begin{Verbatim}[commandchars=\\\{\}]





    \end{Verbatim}

    \begin{Verbatim}[commandchars=\\\{\}]
{\color{incolor}In [{\color{incolor}7}]:} \PY{c+c1}{\PYZsh{} download the file in csv format}
        \PY{o}{!} kaggle datasets download \PYZhy{}f country\PYZus{}vaccinations.csv \PYZhy{}p \PY{l+s+s2}{\PYZdq{}./data\PYZdq{}} gpreda/covid\PYZhy{}world\PYZhy{}vaccination\PYZhy{}progress
        \PY{o}{!} \PY{n+nb}{echo} \PY{l+s+s2}{\PYZdq{}\PYZgt{}\PYZgt{}\PYZgt{} check if data is there\PYZdq{}}
        \PY{o}{!} ls \PY{l+s+s2}{\PYZdq{}./data\PYZdq{}}
\end{Verbatim}


    \begin{Verbatim}[commandchars=\\\{\}]
country\_vaccinations.csv: Skipping, found more recently modified local copy (use --force to force download)
>>> check if data is there
country\_vaccinations.csv

    \end{Verbatim}

    Data wrangling and EDA are the most initial steps in our data science
lifecyle. Most often than not in research, the data is newly collected
or simulated; no one has had time to write up extensive descriptions of
the data. Therefore, an important step is getting to know our data.

    \begin{Verbatim}[commandchars=\\\{\}]
{\color{incolor}In [{\color{incolor}9}]:} \PY{c+c1}{\PYZsh{} read data in}
        \PY{n}{path} \PY{o}{=} \PY{l+s+s2}{\PYZdq{}}\PY{l+s+s2}{./data/}\PY{l+s+s2}{\PYZdq{}}
        \PY{n}{filename} \PY{o}{=} \PY{l+s+s2}{\PYZdq{}}\PY{l+s+s2}{country\PYZus{}vaccinations.csv}\PY{l+s+s2}{\PYZdq{}}
        \PY{n}{read\PYZus{}path} \PY{o}{=} \PY{n}{path} \PY{o}{+} \PY{n}{filename}
        \PY{n}{vax} \PY{o}{=} \PY{n}{pd}\PY{o}{.}\PY{n}{read\PYZus{}csv}\PY{p}{(}\PY{n}{read\PYZus{}path}\PY{p}{)}
        \PY{c+c1}{\PYZsh{} what columns does it have}
        \PY{c+c1}{\PYZsh{} what are data types}
        \PY{n}{vax}\PY{o}{.}\PY{n}{dtypes}
\end{Verbatim}


\begin{Verbatim}[commandchars=\\\{\}]
{\color{outcolor}Out[{\color{outcolor}9}]:} country                                 object
        iso\_code                                object
        date                                    object
        total\_vaccinations                     float64
        people\_vaccinated                      float64
        people\_fully\_vaccinated                float64
        daily\_vaccinations\_raw                 float64
        daily\_vaccinations                     float64
        total\_vaccinations\_per\_hundred         float64
        people\_vaccinated\_per\_hundred          float64
        people\_fully\_vaccinated\_per\_hundred    float64
        daily\_vaccinations\_per\_million         float64
        vaccines                                object
        source\_name                             object
        source\_website                          object
        dtype: object
\end{Verbatim}
            
    Now that we have the data, let's load it in and look at what it has to
offer. To learn more about this dataset, it is often helpful to look at
its \texttt{README} file or just directly go to \texttt{kaggle} and read
the descriptions.
https://www.kaggle.com/gpreda/covid-world-vaccination-progress

The description of columns is usually called a
\texttt{data\ dictionary}. By reading the documentation we learn that
this data is directly sourced from John Hopkins University:
https://github.com/owid/covid-19-data/tree/master/public/data. This is
sometimes important to know because different organizations,
institutions, or even individuals often collect and record data
following different conventions (how are \texttt{NA} values represented,
how are categorical values stored, etc.). You can explore the data
source's description a little more closely:
https://ourworldindata.org/coronavirus-source-data.

    \begin{Verbatim}[commandchars=\\\{\}]
{\color{incolor}In [{\color{incolor}10}]:} \PY{c+c1}{\PYZsh{} how much space does this dataset take?}
         \PY{n}{display}\PY{p}{(}\PY{n}{vax}\PY{o}{.}\PY{n}{memory\PYZus{}usage}\PY{p}{(}\PY{p}{)}\PY{p}{)}
         \PY{c+c1}{\PYZsh{} how many kilobytes?}
         \PY{n+nb}{print}\PY{p}{(}\PY{l+s+s2}{\PYZdq{}}\PY{l+s+si}{\PYZob{}\PYZcb{}}\PY{l+s+s2}{ kB}\PY{l+s+s2}{\PYZdq{}}\PY{o}{.}\PY{n}{format}\PY{p}{(}\PY{n}{np}\PY{o}{.}\PY{n}{round}\PY{p}{(}\PY{n}{vax}\PY{o}{.}\PY{n}{memory\PYZus{}usage}\PY{p}{(}\PY{p}{)}\PY{o}{.}\PY{n}{sum}\PY{p}{(}\PY{p}{)}\PY{o}{/}\PY{l+m+mi}{2}\PY{o}{*}\PY{o}{*}\PY{l+m+mi}{10}\PY{p}{,} \PY{l+m+mi}{2}\PY{p}{)}\PY{p}{)}\PY{p}{)}
\end{Verbatim}


    
    \begin{verbatim}
Index                                    128
country                                35480
iso_code                               35480
date                                   35480
total_vaccinations                     35480
people_vaccinated                      35480
people_fully_vaccinated                35480
daily_vaccinations_raw                 35480
daily_vaccinations                     35480
total_vaccinations_per_hundred         35480
people_vaccinated_per_hundred          35480
people_fully_vaccinated_per_hundred    35480
daily_vaccinations_per_million         35480
vaccines                               35480
source_name                            35480
source_website                         35480
dtype: int64
    \end{verbatim}

    
    \begin{Verbatim}[commandchars=\\\{\}]
519.85 kB

    \end{Verbatim}

    A more comprehensive description of the data types is \texttt{df.info}
function.

    \begin{Verbatim}[commandchars=\\\{\}]
{\color{incolor}In [{\color{incolor}11}]:} \PY{n}{vax}\PY{o}{.}\PY{n}{info}\PY{p}{(}\PY{p}{)}\PY{p}{;}
         \PY{c+c1}{\PYZsh{} vax.dim, vix.shape, vix.dtypes}
\end{Verbatim}


    \begin{Verbatim}[commandchars=\\\{\}]
<class 'pandas.core.frame.DataFrame'>
RangeIndex: 4435 entries, 0 to 4434
Data columns (total 15 columns):
country                                4435 non-null object
iso\_code                               4131 non-null object
date                                   4435 non-null object
total\_vaccinations                     2916 non-null float64
people\_vaccinated                      2483 non-null float64
people\_fully\_vaccinated                1662 non-null float64
daily\_vaccinations\_raw                 2467 non-null float64
daily\_vaccinations                     4281 non-null float64
total\_vaccinations\_per\_hundred         2916 non-null float64
people\_vaccinated\_per\_hundred          2483 non-null float64
people\_fully\_vaccinated\_per\_hundred    1662 non-null float64
daily\_vaccinations\_per\_million         4281 non-null float64
vaccines                               4435 non-null object
source\_name                            4435 non-null object
source\_website                         4435 non-null object
dtypes: float64(9), object(6)
memory usage: 519.9+ KB

    \end{Verbatim}

    \texttt{\textquotesingle{}b\textquotesingle{}\ \ \ \ \ \ \ boolean\ \textquotesingle{}i\textquotesingle{}\ \ \ \ \ \ \ (signed)\ integer\ \textquotesingle{}u\textquotesingle{}\ \ \ \ \ \ \ unsigned\ integer\ \textquotesingle{}f\textquotesingle{}\ \ \ \ \ \ \ floating-point\ \textquotesingle{}c\textquotesingle{}\ \ \ \ \ \ \ complex-floating\ point\ \textquotesingle{}O\textquotesingle{}\ \ \ \ \ \ \ (Python)\ objects\ \textquotesingle{}S\textquotesingle{},\ \textquotesingle{}a\textquotesingle{}\ \ (byte-)string\ \textquotesingle{}U\textquotesingle{}\ \ \ \ \ \ \ Unicode\ \textquotesingle{}V\textquotesingle{}\ \ \ \ \ \ \ raw\ data\ (void)}

    \begin{Verbatim}[commandchars=\\\{\}]
{\color{incolor}In [{\color{incolor}13}]:} \PY{c+c1}{\PYZsh{} what are these \PYZdq{}object\PYZdq{} values, let\PYZsq{}s look at \PYZdq{}country\PYZdq{} for example}
         \PY{n}{countries} \PY{o}{=} \PY{n}{vax}\PY{p}{[}\PY{l+s+s1}{\PYZsq{}}\PY{l+s+s1}{country}\PY{l+s+s1}{\PYZsq{}}\PY{p}{]}
         \PY{n}{countries}\PY{o}{.}\PY{n}{dtype} \PY{c+c1}{\PYZsh{} probably not very helpful}
\end{Verbatim}


\begin{Verbatim}[commandchars=\\\{\}]
{\color{outcolor}Out[{\color{outcolor}13}]:} dtype('O')
\end{Verbatim}
            
    \begin{Verbatim}[commandchars=\\\{\}]
{\color{incolor}In [{\color{incolor}14}]:} \PY{n+nb}{type}\PY{p}{(}\PY{n}{countries}\PY{p}{[}\PY{l+m+mi}{0}\PY{p}{]}\PY{p}{)}
\end{Verbatim}


\begin{Verbatim}[commandchars=\\\{\}]
{\color{outcolor}Out[{\color{outcolor}14}]:} str
\end{Verbatim}
            
    Now that we know what the columns are and (roughly) what they represent,
let's also look at the table as a whole.

\subsubsection{Q: What is the granularity of this
dataset?}\label{q-what-is-the-granularity-of-this-dataset}

Discussion: Is it better to have more granularity, or less granularity?

    \begin{Verbatim}[commandchars=\\\{\}]
{\color{incolor}In [{\color{incolor}15}]:} \PY{n}{vax}\PY{o}{.}\PY{n}{head}\PY{p}{(}\PY{p}{)}
\end{Verbatim}


\begin{Verbatim}[commandchars=\\\{\}]
{\color{outcolor}Out[{\color{outcolor}15}]:}    country iso\_code        date  total\_vaccinations  people\_vaccinated  \textbackslash{}
         0  Albania      ALB  2021-01-10                 0.0                0.0   
         1  Albania      ALB  2021-01-11                 NaN                NaN   
         2  Albania      ALB  2021-01-12               128.0              128.0   
         3  Albania      ALB  2021-01-13               188.0              188.0   
         4  Albania      ALB  2021-01-14               266.0              266.0   
         
            people\_fully\_vaccinated  daily\_vaccinations\_raw  daily\_vaccinations  \textbackslash{}
         0                      NaN                     NaN                 NaN   
         1                      NaN                     NaN                64.0   
         2                      NaN                     NaN                64.0   
         3                      NaN                    60.0                63.0   
         4                      NaN                    78.0                66.0   
         
            total\_vaccinations\_per\_hundred  people\_vaccinated\_per\_hundred  \textbackslash{}
         0                            0.00                           0.00   
         1                             NaN                            NaN   
         2                            0.00                           0.00   
         3                            0.01                           0.01   
         4                            0.01                           0.01   
         
            people\_fully\_vaccinated\_per\_hundred  daily\_vaccinations\_per\_million  \textbackslash{}
         0                                  NaN                             NaN   
         1                                  NaN                            22.0   
         2                                  NaN                            22.0   
         3                                  NaN                            22.0   
         4                                  NaN                            23.0   
         
                   vaccines         source\_name  \textbackslash{}
         0  Pfizer/BioNTech  Ministry of Health   
         1  Pfizer/BioNTech  Ministry of Health   
         2  Pfizer/BioNTech  Ministry of Health   
         3  Pfizer/BioNTech  Ministry of Health   
         4  Pfizer/BioNTech  Ministry of Health   
         
                                               source\_website  
         0  https://shendetesia.gov.al/covid19-ministria-e{\ldots}  
         1  https://shendetesia.gov.al/covid19-ministria-e{\ldots}  
         2  https://shendetesia.gov.al/covid19-ministria-e{\ldots}  
         3  https://shendetesia.gov.al/covid19-ministria-e{\ldots}  
         4  https://shendetesia.gov.al/covid19-ministria-e{\ldots}  
\end{Verbatim}
            
    We have quite a few numerical features, such as
\texttt{people\_vaccinated}, \texttt{daily\_vaccinations} ... We may be
generally interested in some statistics.

    \begin{Verbatim}[commandchars=\\\{\}]
{\color{incolor}In [{\color{incolor}20}]:} \PY{c+c1}{\PYZsh{} compute mean}
         \PY{n+nb}{sum}\PY{p}{(}\PY{n}{vax}\PY{p}{[}\PY{l+s+s1}{\PYZsq{}}\PY{l+s+s1}{daily\PYZus{}vaccinations}\PY{l+s+s1}{\PYZsq{}}\PY{p}{]}\PY{p}{)}\PY{o}{/}\PY{n+nb}{len}\PY{p}{(}\PY{n}{vax}\PY{p}{[}\PY{l+s+s1}{\PYZsq{}}\PY{l+s+s1}{daily\PYZus{}vaccinations}\PY{l+s+s1}{\PYZsq{}}\PY{p}{]}\PY{p}{)} \PY{c+c1}{\PYZsh{} comments?}
         \PY{n}{vax}\PY{p}{[}\PY{l+s+s1}{\PYZsq{}}\PY{l+s+s1}{daily\PYZus{}vaccinations}\PY{l+s+s1}{\PYZsq{}}\PY{p}{]}\PY{o}{.}\PY{n}{mean}\PY{p}{(}\PY{p}{)}
\end{Verbatim}


\begin{Verbatim}[commandchars=\\\{\}]
{\color{outcolor}Out[{\color{outcolor}20}]:} 55316.880168185
\end{Verbatim}
            
    We have some \texttt{NaN} values, this can be due to a few different
reasons depending on context. But remember: \textbf{If there is no data,
it does not mean that there is no problem}.

\subsubsection{\texorpdfstring{Q: Is getting rid of data points that
contain \texttt{NaN}'s a good
idea?}{Q: Is getting rid of data points that contain NaN's a good idea?}}\label{q-is-getting-rid-of-data-points-that-contain-nans-a-good-idea}

    Now that we know in general: * the (physical) size of data * the
dimensions of data * what each row / column represents * the data types
contained in this data * analmolies

Now we can dive into the data values themselves and find out what
properties this dataset has.

    \begin{Verbatim}[commandchars=\\\{\}]
{\color{incolor}In [{\color{incolor}24}]:} \PY{c+c1}{\PYZsh{} what happens if we get rid of NaN\PYZsq{}s}
         \PY{n}{vax\PYZus{}clean} \PY{o}{=} \PY{n}{vax}\PY{o}{.}\PY{n}{dropna}\PY{p}{(}\PY{p}{)}
         \PY{n}{vax\PYZus{}clean} \PY{o}{=} \PY{n}{vax\PYZus{}clean}\PY{o}{.}\PY{n}{sort\PYZus{}values}\PY{p}{(}\PY{n}{by}\PY{o}{=}\PY{l+s+s2}{\PYZdq{}}\PY{l+s+s2}{date}\PY{l+s+s2}{\PYZdq{}}\PY{p}{,} \PY{n}{ascending}\PY{o}{=}\PY{k+kc}{False}\PY{p}{)}\PY{o}{.}\PY{n}{reset\PYZus{}index}\PY{p}{(}\PY{p}{)}
         \PY{n}{display}\PY{p}{(}\PY{n}{vax\PYZus{}clean}\PY{o}{.}\PY{n}{head}\PY{p}{(}\PY{p}{)}\PY{p}{)}
         \PY{n+nb}{print}\PY{p}{(}\PY{n}{vax\PYZus{}clean}\PY{o}{.}\PY{n}{shape}\PY{p}{)}
         
         \PY{c+c1}{\PYZsh{} so getting rid of NaN\PYZsq{}s may not always be the best idea}
\end{Verbatim}


    
    \begin{verbatim}
   index        country iso_code        date  total_vaccinations  \
0   4342  United States      USA  2021-02-27          72806180.0   
1   3417        Romania      ROU  2021-02-27           1521737.0   
2    580         Brazil      BRA  2021-02-27           8322042.0   
3   1981      Indonesia      IDN  2021-02-27           2598535.0   
4   2794        Morocco      MAR  2021-02-27           3597903.0   

   people_vaccinated  people_fully_vaccinated  daily_vaccinations_raw  \
0         48435536.0               23698627.0               2352116.0   
1           905142.0                 616595.0                 15704.0   
2          6437836.0                1884206.0                220255.0   
3          1616165.0                 982370.0                149084.0   
4          3435997.0                 161906.0                173608.0   

   daily_vaccinations  total_vaccinations_per_hundred  \
0           1645240.0                           21.77   
1             24351.0                            7.91   
2            215553.0                            3.92   
3             91687.0                            0.95   
4            162387.0                            9.75   

   people_vaccinated_per_hundred  people_fully_vaccinated_per_hundred  \
0                          14.48                                 7.09   
1                           4.71                                 3.21   
2                           3.03                                 0.89   
3                           0.59                                 0.36   
4                           9.31                                 0.44   

   daily_vaccinations_per_million  \
0                          4919.0   
1                          1266.0   
2                          1014.0   
3                           335.0   
4                          4399.0   

                                       vaccines  \
0                      Moderna, Pfizer/BioNTech   
1  Moderna, Oxford/AstraZeneca, Pfizer/BioNTech   
2                   Oxford/AstraZeneca, Sinovac   
3                                       Sinovac   
4         Oxford/AstraZeneca, Sinopharm/Beijing   

                                   source_name  \
0   Centers for Disease Control and Prevention   
1                        Government of Romania   
2  Regional governments via Coronavirus Brasil   
3                           Ministry of Health   
4                           Ministry of Health   

                                      source_website  
0  https://covid.cdc.gov/covid-data-tracker/#vacc...  
1  https://vaccinare-covid.gov.ro/wp-content/uplo...  
2                 https://coronavirusbra1.github.io/  
3                          https://www.kemkes.go.id/  
4  http://www.covidmaroc.ma/Documents/BULLETIN/27...  
    \end{verbatim}

    
    \begin{Verbatim}[commandchars=\\\{\}]
(1316, 16)

    \end{Verbatim}

    How many countries are represented?

    \begin{Verbatim}[commandchars=\\\{\}]
{\color{incolor}In [{\color{incolor}28}]:} \PY{n}{np}\PY{o}{.}\PY{n}{unique}\PY{p}{(}\PY{n}{vax}\PY{p}{[}\PY{l+s+s1}{\PYZsq{}}\PY{l+s+s1}{country}\PY{l+s+s1}{\PYZsq{}}\PY{p}{]}\PY{p}{)}
\end{Verbatim}


\begin{Verbatim}[commandchars=\\\{\}]
{\color{outcolor}Out[{\color{outcolor}28}]:} array(['Albania', 'Algeria', 'Andorra', 'Anguilla', 'Argentina',
                'Australia', 'Austria', 'Azerbaijan', 'Bahrain', 'Bangladesh',
                'Barbados', 'Belarus', 'Belgium', 'Bermuda', 'Bolivia', 'Brazil',
                'Bulgaria', 'Cambodia', 'Canada', 'Cayman Islands', 'Chile',
                'China', 'Colombia', 'Costa Rica', 'Croatia', 'Cyprus', 'Czechia',
                'Denmark', 'Dominican Republic', 'Ecuador', 'Egypt', 'El Salvador',
                'England', 'Estonia', 'Faeroe Islands', 'Falkland Islands',
                'Finland', 'France', 'Germany', 'Gibraltar', 'Greece', 'Greenland',
                'Guernsey', 'Guyana', 'Hungary', 'Iceland', 'India', 'Indonesia',
                'Iran', 'Ireland', 'Isle of Man', 'Israel', 'Italy', 'Japan',
                'Jersey', 'Kazakhstan', 'Kuwait', 'Latvia', 'Lebanon',
                'Liechtenstein', 'Lithuania', 'Luxembourg', 'Macao', 'Maldives',
                'Malta', 'Mauritius', 'Mexico', 'Monaco', 'Montenegro', 'Morocco',
                'Myanmar', 'Nepal', 'Netherlands', 'New Zealand',
                'Northern Cyprus', 'Northern Ireland', 'Norway', 'Oman',
                'Pakistan', 'Panama', 'Paraguay', 'Peru', 'Poland', 'Portugal',
                'Qatar', 'Romania', 'Russia', 'Saint Helena', 'Saudi Arabia',
                'Scotland', 'Senegal', 'Serbia', 'Seychelles', 'Singapore',
                'Slovakia', 'Slovenia', 'South Africa', 'South Korea', 'Spain',
                'Sri Lanka', 'Sweden', 'Switzerland', 'Trinidad and Tobago',
                'Turkey', 'Turks and Caicos Islands', 'Ukraine',
                'United Arab Emirates', 'United Kingdom', 'United States',
                'Venezuela', 'Wales', 'Zimbabwe'], dtype=object)
\end{Verbatim}
            
    How much data do we have on each country, are they equal?

    \begin{Verbatim}[commandchars=\\\{\}]
{\color{incolor}In [{\color{incolor}30}]:} \PY{n}{vax}\PY{p}{[}\PY{l+s+s1}{\PYZsq{}}\PY{l+s+s1}{country}\PY{l+s+s1}{\PYZsq{}}\PY{p}{]}\PY{o}{.}\PY{n}{value\PYZus{}counts}\PY{p}{(}\PY{p}{)}
\end{Verbatim}


\begin{Verbatim}[commandchars=\\\{\}]
{\color{outcolor}Out[{\color{outcolor}30}]:} Lithuania         82
         Scotland          76
         United Kingdom    76
         Wales             76
         England           76
                           ..
         Senegal            5
         South Korea        3
         Ukraine            3
         Saint Helena       1
         Greenland          1
         Name: country, Length: 112, dtype: int64
\end{Verbatim}
            
    What is the range of dates?

    \begin{Verbatim}[commandchars=\\\{\}]
{\color{incolor}In [{\color{incolor}33}]:} \PY{c+c1}{\PYZsh{} these achieves the same goal}
         \PY{n+nb}{print}\PY{p}{(}\PY{l+s+s2}{\PYZdq{}}\PY{l+s+s2}{the dates are from }\PY{l+s+si}{\PYZob{}\PYZcb{}}\PY{l+s+s2}{ to }\PY{l+s+si}{\PYZob{}\PYZcb{}}\PY{l+s+s2}{\PYZdq{}}\PY{o}{.}\PY{n}{format}\PY{p}{(}\PY{n}{np}\PY{o}{.}\PY{n}{amin}\PY{p}{(}\PY{n}{vax}\PY{p}{[}\PY{l+s+s1}{\PYZsq{}}\PY{l+s+s1}{date}\PY{l+s+s1}{\PYZsq{}}\PY{p}{]}\PY{p}{)}\PY{p}{,} \PY{n}{np}\PY{o}{.}\PY{n}{amax}\PY{p}{(}\PY{n}{vax}\PY{p}{[}\PY{l+s+s1}{\PYZsq{}}\PY{l+s+s1}{date}\PY{l+s+s1}{\PYZsq{}}\PY{p}{]}\PY{p}{)}\PY{p}{)}\PY{p}{)}
         \PY{c+c1}{\PYZsh{} or you can do this}
         \PY{p}{(}\PY{n}{vax}\PY{p}{[}\PY{l+s+s1}{\PYZsq{}}\PY{l+s+s1}{date}\PY{l+s+s1}{\PYZsq{}}\PY{p}{]}\PY{o}{.}\PY{n}{min}\PY{p}{(}\PY{p}{)}\PY{p}{,} \PY{n}{vax}\PY{p}{[}\PY{l+s+s1}{\PYZsq{}}\PY{l+s+s1}{date}\PY{l+s+s1}{\PYZsq{}}\PY{p}{]}\PY{o}{.}\PY{n}{max}\PY{p}{(}\PY{p}{)}\PY{p}{)}
         \PY{c+c1}{\PYZsh{} so, about 3 months worth of data}
\end{Verbatim}


    \begin{Verbatim}[commandchars=\\\{\}]
the dates are from 2020-12-08 to 2021-02-27

    \end{Verbatim}

\begin{Verbatim}[commandchars=\\\{\}]
{\color{outcolor}Out[{\color{outcolor}33}]:} ('2020-12-08', '2021-02-27')
\end{Verbatim}
            
    How is the world vaccination progressing? Namely, on average, how many
people get vaccinated everyday?

    \begin{Verbatim}[commandchars=\\\{\}]
{\color{incolor}In [{\color{incolor}34}]:} \PY{c+c1}{\PYZsh{} these achieves the same goal}
         \PY{n+nb}{print}\PY{p}{(}\PY{l+s+s2}{\PYZdq{}}\PY{l+s+s2}{daily vaccination average: }\PY{l+s+si}{\PYZob{}\PYZcb{}}\PY{l+s+s2}{\PYZdq{}}\PY{o}{.}\PY{n}{format}\PY{p}{(}\PY{n}{vax}\PY{p}{[}\PY{l+s+s1}{\PYZsq{}}\PY{l+s+s1}{daily\PYZus{}vaccinations}\PY{l+s+s1}{\PYZsq{}}\PY{p}{]}\PY{o}{.}\PY{n}{mean}\PY{p}{(}\PY{p}{)}\PY{p}{)}\PY{p}{)}
         \PY{c+c1}{\PYZsh{} or you can do this}
         \PY{n+nb}{print}\PY{p}{(}\PY{l+s+s2}{\PYZdq{}}\PY{l+s+s2}{daily vaccination average: }\PY{l+s+si}{\PYZob{}\PYZcb{}}\PY{l+s+s2}{\PYZdq{}}\PY{o}{.}\PY{n}{format}\PY{p}{(}\PY{n}{np}\PY{o}{.}\PY{n}{mean}\PY{p}{(}\PY{n}{vax}\PY{p}{[}\PY{l+s+s1}{\PYZsq{}}\PY{l+s+s1}{daily\PYZus{}vaccinations}\PY{l+s+s1}{\PYZsq{}}\PY{p}{]}\PY{p}{)}\PY{p}{)}\PY{p}{)}
         \PY{c+c1}{\PYZsh{} that\PYZsq{}s a bit slow, but we are making progress}
\end{Verbatim}


    \begin{Verbatim}[commandchars=\\\{\}]
daily vaccination average: 55316.880168185
daily vaccination average: 55316.880168185

    \end{Verbatim}

    \begin{Verbatim}[commandchars=\\\{\}]
{\color{incolor}In [{\color{incolor} }]:} \PY{c+c1}{\PYZsh{} you can do numerical computations on pd.Series directly}
        \PY{n}{vax}\PY{p}{[}\PY{l+s+s1}{\PYZsq{}}\PY{l+s+s1}{people\PYZus{}fully\PYZus{}vaccinated}\PY{l+s+s1}{\PYZsq{}}\PY{p}{]} \PY{o}{/} \PY{n}{vax}\PY{p}{[}\PY{l+s+s1}{\PYZsq{}}\PY{l+s+s1}{\PYZsq{}}\PY{p}{]}
\end{Verbatim}


    Here is a quick way: \texttt{df.describe} gives you some quick
statistics of your numerical data. It has a few advantages, but need to
be careful about interpretability.

    \begin{Verbatim}[commandchars=\\\{\}]
{\color{incolor}In [{\color{incolor}35}]:} \PY{n}{vax}\PY{o}{.}\PY{n}{describe}\PY{p}{(}\PY{p}{)}
\end{Verbatim}


\begin{Verbatim}[commandchars=\\\{\}]
{\color{outcolor}Out[{\color{outcolor}35}]:}        total\_vaccinations  people\_vaccinated  people\_fully\_vaccinated  \textbackslash{}
         count        2.916000e+03       2.483000e+03             1.662000e+03   
         mean         1.709487e+06       1.481442e+06             4.888581e+05   
         std          5.774372e+06       4.646374e+06             1.899838e+06   
         min          0.000000e+00       0.000000e+00             1.000000e+00   
         25\%          3.154575e+04       2.799900e+04             1.119425e+04   
         50\%          2.049345e+05       1.822800e+05             5.062800e+04   
         75\%          8.565680e+05       7.471645e+05             2.607428e+05   
         max          7.280618e+07       4.843554e+07             2.369863e+07   
         
                daily\_vaccinations\_raw  daily\_vaccinations  \textbackslash{}
         count            2.467000e+03        4.281000e+03   
         mean             7.517774e+04        5.531688e+04   
         std              2.111072e+05        1.744120e+05   
         min             -5.001200e+04        1.000000e+00   
         25\%              2.282000e+03        1.121000e+03   
         50\%              1.183300e+04        5.857000e+03   
         75\%              5.366500e+04        2.704700e+04   
         max              2.352116e+06        1.916190e+06   
         
                total\_vaccinations\_per\_hundred  people\_vaccinated\_per\_hundred  \textbackslash{}
         count                     2916.000000                    2483.000000   
         mean                         7.078261                       5.751832   
         std                         13.147480                       9.446641   
         min                          0.000000                       0.000000   
         25\%                          0.620000                       0.600000   
         50\%                          2.735000                       2.530000   
         75\%                          6.675000                       5.200000   
         max                        106.530000                      67.410000   
         
                people\_fully\_vaccinated\_per\_hundred  daily\_vaccinations\_per\_million  
         count                          1662.000000                     4281.000000  
         mean                              2.262515                     2404.288951  
         std                               5.501138                     4378.201585  
         min                               0.000000                        0.000000  
         25\%                               0.212500                      321.000000  
         50\%                               0.840000                     1064.000000  
         75\%                               1.935000                     2190.000000  
         max                              39.110000                    54264.000000  
\end{Verbatim}
            
    Now we understand better the numerical properties of our data. We can
start to ask some more complex questions.

    \subsection{Q: Who's not vaccinated?}\label{q-whos-not-vaccinated}

    \[
\text{(Total number of people vaccinated per hundred) } =\frac{(\text{Total number of people fully vaccinated})}{(\text{Total population up to the date in the country})} \times 100\%
\]

    \begin{Verbatim}[commandchars=\\\{\}]
{\color{incolor}In [{\color{incolor}37}]:} \PY{n}{vax}\PY{p}{[}\PY{l+s+s1}{\PYZsq{}}\PY{l+s+s1}{total\PYZus{}population}\PY{l+s+s1}{\PYZsq{}}\PY{p}{]} \PY{o}{=} \PY{n}{vax}\PY{p}{[}\PY{l+s+s1}{\PYZsq{}}\PY{l+s+s1}{people\PYZus{}fully\PYZus{}vaccinated}\PY{l+s+s1}{\PYZsq{}}\PY{p}{]} \PY{o}{/} \PY{p}{(}\PY{n}{vax}\PY{p}{[}\PY{l+s+s1}{\PYZsq{}}\PY{l+s+s1}{total\PYZus{}vaccinations\PYZus{}per\PYZus{}hundred}\PY{l+s+s1}{\PYZsq{}}\PY{p}{]} \PY{o}{/} \PY{l+m+mi}{100}\PY{p}{)}
         \PY{n}{vax}\PY{p}{[}\PY{p}{[}\PY{l+s+s1}{\PYZsq{}}\PY{l+s+s1}{country}\PY{l+s+s1}{\PYZsq{}}\PY{p}{,} \PY{l+s+s1}{\PYZsq{}}\PY{l+s+s1}{total\PYZus{}population}\PY{l+s+s1}{\PYZsq{}}\PY{p}{]}\PY{p}{]}\PY{o}{.}\PY{n}{dropna}\PY{p}{(}\PY{p}{)}
\end{Verbatim}


\begin{Verbatim}[commandchars=\\\{\}]
{\color{outcolor}Out[{\color{outcolor}37}]:}       country  total\_population
         23    Albania      5.000000e+03
         30    Albania      1.095000e+06
         38    Albania      1.018333e+06
         39    Albania      5.554545e+05
         43    Albania      2.847826e+05
         {\ldots}       {\ldots}               {\ldots}
         4420    Wales      1.705384e+05
         4421    Wales      1.993241e+05
         4422    Wales      2.296220e+05
         4423    Wales      2.569384e+05
         4424    Wales      2.792505e+05
         
         [1654 rows x 2 columns]
\end{Verbatim}
            
    Now we can see who's not vaccinated in each country.

    \begin{Verbatim}[commandchars=\\\{\}]
{\color{incolor}In [{\color{incolor}38}]:} \PY{n}{vax}\PY{p}{[}\PY{l+s+s1}{\PYZsq{}}\PY{l+s+s1}{people\PYZus{}unvaccinated}\PY{l+s+s1}{\PYZsq{}}\PY{p}{]} \PY{o}{=} \PY{n}{vax}\PY{p}{[}\PY{l+s+s1}{\PYZsq{}}\PY{l+s+s1}{total\PYZus{}population}\PY{l+s+s1}{\PYZsq{}}\PY{p}{]} \PY{o}{\PYZhy{}} \PY{n}{vax}\PY{p}{[}\PY{l+s+s1}{\PYZsq{}}\PY{l+s+s1}{people\PYZus{}fully\PYZus{}vaccinated}\PY{l+s+s1}{\PYZsq{}}\PY{p}{]}
         \PY{n}{vax}\PY{p}{[}\PY{p}{[}\PY{l+s+s1}{\PYZsq{}}\PY{l+s+s1}{country}\PY{l+s+s1}{\PYZsq{}}\PY{p}{,} \PY{l+s+s1}{\PYZsq{}}\PY{l+s+s1}{people\PYZus{}unvaccinated}\PY{l+s+s1}{\PYZsq{}}\PY{p}{]}\PY{p}{]}\PY{o}{.}\PY{n}{dropna}\PY{p}{(}\PY{p}{)}
\end{Verbatim}


\begin{Verbatim}[commandchars=\\\{\}]
{\color{outcolor}Out[{\color{outcolor}38}]:}       country  people\_unvaccinated
         23    Albania         4.999000e+03
         30    Albania         1.094562e+06
         38    Albania         1.017722e+06
         39    Albania         5.548435e+05
         43    Albania         2.841276e+05
         {\ldots}       {\ldots}                  {\ldots}
         4420    Wales         1.208094e+05
         4421    Wales         1.400451e+05
         4422    Wales         1.597710e+05
         4423    Wales         1.768764e+05
         4424    Wales         1.901975e+05
         
         [1654 rows x 2 columns]
\end{Verbatim}
            
    We can even ask further questions as to which country is most recently,
most vaccinated, and most un-vaccinated?

    \begin{Verbatim}[commandchars=\\\{\}]
{\color{incolor}In [{\color{incolor}39}]:} \PY{c+c1}{\PYZsh{} let\PYZsq{}s move to the same recorded day}
         \PY{n}{vax\PYZus{}curr} \PY{o}{=} \PY{n}{vax}\PY{p}{[}\PY{n}{vax}\PY{p}{[}\PY{l+s+s1}{\PYZsq{}}\PY{l+s+s1}{date}\PY{l+s+s1}{\PYZsq{}}\PY{p}{]} \PY{o}{==} \PY{l+s+s1}{\PYZsq{}}\PY{l+s+s1}{2021\PYZhy{}02\PYZhy{}26}\PY{l+s+s1}{\PYZsq{}}\PY{p}{]}
\end{Verbatim}


    \begin{Verbatim}[commandchars=\\\{\}]
{\color{incolor}In [{\color{incolor}40}]:} \PY{n}{vax}\PY{o}{.}\PY{n}{iloc}\PY{p}{[}\PY{n}{vax\PYZus{}curr}\PY{p}{[}\PY{l+s+s1}{\PYZsq{}}\PY{l+s+s1}{people\PYZus{}fully\PYZus{}vaccinated}\PY{l+s+s1}{\PYZsq{}}\PY{p}{]}\PY{o}{.}\PY{n}{idxmax}\PY{p}{]}
\end{Verbatim}


\begin{Verbatim}[commandchars=\\\{\}]
{\color{outcolor}Out[{\color{outcolor}40}]:} country                                                                    United States
         iso\_code                                                                             USA
         date                                                                          2021-02-26
         total\_vaccinations                                                           7.04541e+07
         people\_vaccinated                                                            4.71842e+07
         people\_fully\_vaccinated                                                      2.26134e+07
         daily\_vaccinations\_raw                                                       2.17995e+06
         daily\_vaccinations                                                           1.55272e+06
         total\_vaccinations\_per\_hundred                                                     21.07
         people\_vaccinated\_per\_hundred                                                      14.11
         people\_fully\_vaccinated\_per\_hundred                                                 6.76
         daily\_vaccinations\_per\_million                                                      4643
         vaccines                                                        Moderna, Pfizer/BioNTech
         source\_name                                   Centers for Disease Control and Prevention
         source\_website                         https://covid.cdc.gov/covid-data-tracker/\#vacc{\ldots}
         total\_population                                                             1.07325e+08
         people\_unvaccinated                                                          8.47116e+07
         Name: 4341, dtype: object
\end{Verbatim}
            
    \begin{Verbatim}[commandchars=\\\{\}]
{\color{incolor}In [{\color{incolor}41}]:} \PY{n}{vax}\PY{o}{.}\PY{n}{iloc}\PY{p}{[}\PY{n}{vax\PYZus{}curr}\PY{p}{[}\PY{l+s+s1}{\PYZsq{}}\PY{l+s+s1}{people\PYZus{}fully\PYZus{}vaccinated}\PY{l+s+s1}{\PYZsq{}}\PY{p}{]}\PY{o}{.}\PY{n}{idxmin}\PY{p}{]}
\end{Verbatim}


\begin{Verbatim}[commandchars=\\\{\}]
{\color{outcolor}Out[{\color{outcolor}41}]:} country                                                                      Isle of Man
         iso\_code                                                                             IMN
         date                                                                          2021-02-26
         total\_vaccinations                                                                 19884
         people\_vaccinated                                                                  13600
         people\_fully\_vaccinated                                                             6284
         daily\_vaccinations\_raw                                                              1089
         daily\_vaccinations                                                                   391
         total\_vaccinations\_per\_hundred                                                     23.38
         people\_vaccinated\_per\_hundred                                                      15.99
         people\_fully\_vaccinated\_per\_hundred                                                 7.39
         daily\_vaccinations\_per\_million                                                      4598
         vaccines                                             Oxford/AstraZeneca, Pfizer/BioNTech
         source\_name                                                       Isle of Man Government
         source\_website                         https://covid19.gov.im/general-information/cov{\ldots}
         total\_population                                                                 26877.7
         people\_unvaccinated                                                              20593.7
         Name: 2084, dtype: object
\end{Verbatim}
            
    There are many more topics we can explore (feel free to try to answer
these on your own: How effective are vaccines? What's the busiest day,
is there a pattern?). Notice that all we are doing are just computing
simple statistics, but the key is to learn about our data in that: (1)
get familiar with manipulating this dataset and (2) explore the
\textbf{scope} and \textbf{temporality} of this dataset.

    \section{Visualizing Our Data}\label{visualizing-our-data}

Depending on our needs and whether the data is categorical / numerical,
we can have different ways to look at data. We can understand the
dataset in a much more direct and intuitive way by visualizing. We will
discuss more in the upcoming lectures.

Today, we will be working with \texttt{datascience} which provides some
useful visualization tools, and see a few examples of more standard
packages such as \texttt{pandas}, \texttt{matplotlib} and
\texttt{seaborn}.

    \subsection{Q: Is the vaccination rate in the United States looking
up?}\label{q-is-the-vaccination-rate-in-the-united-states-looking-up}

    \begin{Verbatim}[commandchars=\\\{\}]
{\color{incolor}In [{\color{incolor}43}]:} \PY{c+c1}{\PYZsh{} focus on the US}
         \PY{n}{us\PYZus{}vax} \PY{o}{=} \PY{n}{vax}\PY{p}{[}\PY{n}{vax}\PY{p}{[}\PY{l+s+s1}{\PYZsq{}}\PY{l+s+s1}{country}\PY{l+s+s1}{\PYZsq{}}\PY{p}{]} \PY{o}{==} \PY{l+s+s2}{\PYZdq{}}\PY{l+s+s2}{United States}\PY{l+s+s2}{\PYZdq{}}\PY{p}{]}
         \PY{n}{us\PYZus{}vax}\PY{o}{.}\PY{n}{shape}
\end{Verbatim}


\begin{Verbatim}[commandchars=\\\{\}]
{\color{outcolor}Out[{\color{outcolor}43}]:} (70, 17)
\end{Verbatim}
            
    What is a good way to understand the temporal trend of a numerical
value?

    \begin{Verbatim}[commandchars=\\\{\}]
{\color{incolor}In [{\color{incolor}44}]:} \PY{n}{us\PYZus{}daily\PYZus{}trend} \PY{o}{=} \PY{n}{Table}\PY{p}{(}\PY{p}{)}\PY{o}{.}\PY{n}{with\PYZus{}columns}\PY{p}{(}\PY{p}{[}
             \PY{l+s+s1}{\PYZsq{}}\PY{l+s+s1}{date}\PY{l+s+s1}{\PYZsq{}}\PY{p}{,} \PY{n}{us\PYZus{}vax}\PY{p}{[}\PY{l+s+s1}{\PYZsq{}}\PY{l+s+s1}{date}\PY{l+s+s1}{\PYZsq{}}\PY{p}{]}\PY{p}{,} 
             \PY{l+s+s1}{\PYZsq{}}\PY{l+s+s1}{daily\PYZus{}vaccinations}\PY{l+s+s1}{\PYZsq{}}\PY{p}{,} \PY{n}{us\PYZus{}vax}\PY{p}{[}\PY{l+s+s1}{\PYZsq{}}\PY{l+s+s1}{daily\PYZus{}vaccinations}\PY{l+s+s1}{\PYZsq{}}\PY{p}{]}
         \PY{p}{]}\PY{p}{)}
         
         \PY{c+c1}{\PYZsh{} plot}
         \PY{c+c1}{\PYZsh{}help(us\PYZus{}daily\PYZus{}trend.plot)}
         \PY{n}{us\PYZus{}daily\PYZus{}trend}\PY{o}{.}\PY{n}{plot}\PY{p}{(}\PY{l+s+s1}{\PYZsq{}}\PY{l+s+s1}{date}\PY{l+s+s1}{\PYZsq{}}\PY{p}{)}\PY{p}{;}
         \PY{n}{plt}\PY{o}{.}\PY{n}{xticks}\PY{p}{(}\PY{n}{rotation} \PY{o}{=} \PY{l+m+mi}{45}\PY{p}{)}\PY{p}{;}
\end{Verbatim}


    \begin{center}
    \adjustimage{max size={0.9\linewidth}{0.9\paperheight}}{output_48_0.png}
    \end{center}
    { \hspace*{\fill} \\}
    
    How to do this in \texttt{pandas}? Look at \texttt{pd.DataFrame().plot}.

    \begin{Verbatim}[commandchars=\\\{\}]
{\color{incolor}In [{\color{incolor}49}]:} \PY{n}{us\PYZus{}vax}\PY{o}{.}\PY{n}{plot}\PY{p}{(}\PY{l+s+s1}{\PYZsq{}}\PY{l+s+s1}{date}\PY{l+s+s1}{\PYZsq{}}\PY{p}{,} \PY{l+s+s1}{\PYZsq{}}\PY{l+s+s1}{daily\PYZus{}vaccinations}\PY{l+s+s1}{\PYZsq{}}\PY{p}{,} \PY{n}{rot}\PY{o}{=}\PY{l+m+mi}{90}\PY{p}{)}\PY{p}{;}
\end{Verbatim}


    \begin{center}
    \adjustimage{max size={0.9\linewidth}{0.9\paperheight}}{output_50_0.png}
    \end{center}
    { \hspace*{\fill} \\}
    
    We can get even more detailed plots with \texttt{matplotlib}.

    Here are a few \texttt{stackexchange} posts I consulted. Most often than
not, your question has been answered.

\begin{itemize}
\tightlist
\item
  https://stackoverflow.com/questions/33382619/plot-a-horizontal-line-using-matplotlib
\item
  https://stackoverflow.com/questions/23248435/fill-between-two-vertical-lines-in-matplotlib
\item
  https://stackoverflow.com/questions/18089667/how-to-estimate-how-much-memory-a-pandas-dataframe-will-need/47751572
\item
  https://stackoverflow.com/questions/20625582/how-to-deal-with-settingwithcopywarning-in-pandas
\end{itemize}

The documentations are also good places to go to, they usually contain
useful examples.

    \begin{Verbatim}[commandchars=\\\{\}]
{\color{incolor}In [{\color{incolor}53}]:} \PY{n}{plt}\PY{o}{.}\PY{n}{figure}\PY{p}{(}\PY{n}{figsize}\PY{o}{=}\PY{p}{(}\PY{l+m+mi}{10}\PY{p}{,} \PY{l+m+mi}{8}\PY{p}{)}\PY{p}{)}
         \PY{n}{plt}\PY{o}{.}\PY{n}{xlabel}\PY{p}{(}\PY{l+s+s1}{\PYZsq{}}\PY{l+s+s1}{date}\PY{l+s+s1}{\PYZsq{}}\PY{p}{)}\PY{p}{;} \PY{n}{plt}\PY{o}{.}\PY{n}{ylabel}\PY{p}{(}\PY{l+s+s1}{\PYZsq{}}\PY{l+s+s1}{number of vaccinations}\PY{l+s+s1}{\PYZsq{}}\PY{p}{)}\PY{p}{;} \PY{n}{plt}\PY{o}{.}\PY{n}{title}\PY{p}{(}\PY{l+s+s1}{\PYZsq{}}\PY{l+s+s1}{daily vaccination trend}\PY{l+s+s1}{\PYZsq{}}\PY{p}{)}\PY{p}{;} 
         \PY{c+c1}{\PYZsh{} change date to datetime}
         \PY{n}{plt}\PY{o}{.}\PY{n}{plot}\PY{p}{(}\PY{n}{pd}\PY{o}{.}\PY{n}{to\PYZus{}datetime}\PY{p}{(}\PY{n}{us\PYZus{}vax}\PY{p}{[}\PY{l+s+s1}{\PYZsq{}}\PY{l+s+s1}{date}\PY{l+s+s1}{\PYZsq{}}\PY{p}{]}\PY{p}{,} \PY{n+nb}{format} \PY{o}{=} \PY{l+s+s1}{\PYZsq{}}\PY{l+s+s1}{\PYZpc{}}\PY{l+s+s1}{Y\PYZhy{}}\PY{l+s+s1}{\PYZpc{}}\PY{l+s+s1}{m\PYZhy{}}\PY{l+s+si}{\PYZpc{}d}\PY{l+s+s1}{\PYZsq{}}\PY{p}{)}\PY{o}{.}\PY{n}{dt}\PY{o}{.}\PY{n}{date}\PY{p}{,} \PY{n}{us\PYZus{}vax}\PY{p}{[}\PY{l+s+s1}{\PYZsq{}}\PY{l+s+s1}{daily\PYZus{}vaccinations}\PY{l+s+s1}{\PYZsq{}}\PY{p}{]}\PY{p}{,} \PYZbs{}
                  \PY{n}{marker}\PY{o}{=}\PY{l+s+s1}{\PYZsq{}}\PY{l+s+s1}{o}\PY{l+s+s1}{\PYZsq{}}\PY{p}{,} \PY{n}{label}\PY{o}{=}\PY{l+s+s1}{\PYZsq{}}\PY{l+s+s1}{trend}\PY{l+s+s1}{\PYZsq{}}\PY{p}{)}\PY{p}{;}
         \PY{n}{plt}\PY{o}{.}\PY{n}{xticks}\PY{p}{(}\PY{n}{rotation} \PY{o}{=} \PY{l+m+mi}{45}\PY{p}{)}\PY{p}{;} \PY{n}{plt}\PY{o}{.}\PY{n}{legend}\PY{p}{(}\PY{n}{loc}\PY{o}{=}\PY{l+s+s1}{\PYZsq{}}\PY{l+s+s1}{upper left}\PY{l+s+s1}{\PYZsq{}}\PY{p}{)}\PY{p}{;} \PY{n}{plt}\PY{o}{.}\PY{n}{grid}\PY{p}{(}\PY{p}{)}\PY{p}{;}
         \PY{n}{plt}\PY{o}{.}\PY{n}{axhline}\PY{p}{(}\PY{n}{np}\PY{o}{.}\PY{n}{mean}\PY{p}{(}\PY{n}{us\PYZus{}vax}\PY{p}{[}\PY{l+s+s1}{\PYZsq{}}\PY{l+s+s1}{daily\PYZus{}vaccinations}\PY{l+s+s1}{\PYZsq{}}\PY{p}{]}\PY{p}{)}\PY{p}{,} \PY{n}{color}\PY{o}{=}\PY{l+s+s1}{\PYZsq{}}\PY{l+s+s1}{green}\PY{l+s+s1}{\PYZsq{}}\PY{p}{,} \PY{n}{lw}\PY{o}{=}\PY{l+m+mi}{2}\PY{p}{,} \PY{n}{label}\PY{o}{=}\PY{l+s+s2}{\PYZdq{}}\PY{l+s+s2}{mean level}\PY{l+s+s2}{\PYZdq{}}\PY{p}{)}\PY{p}{;}
         
         \PY{n}{plt}\PY{o}{.}\PY{n}{fill\PYZus{}betweenx}\PY{p}{(}\PY{n}{us\PYZus{}vax}\PY{p}{[}\PY{l+s+s1}{\PYZsq{}}\PY{l+s+s1}{daily\PYZus{}vaccinations}\PY{l+s+s1}{\PYZsq{}}\PY{p}{]}\PY{p}{,} \PY{n+nb}{sorted}\PY{p}{(}\PY{n+nb}{list}\PY{p}{(}\PY{n}{us\PYZus{}vax}\PY{p}{[}\PY{l+s+s1}{\PYZsq{}}\PY{l+s+s1}{date}\PY{l+s+s1}{\PYZsq{}}\PY{p}{]}\PY{p}{)}\PY{p}{)}\PY{p}{[}\PY{l+m+mi}{30}\PY{p}{]}\PY{p}{,} \PYZbs{}
                           \PY{n+nb}{sorted}\PY{p}{(}\PY{n+nb}{list}\PY{p}{(}\PY{n}{us\PYZus{}vax}\PY{p}{[}\PY{l+s+s1}{\PYZsq{}}\PY{l+s+s1}{date}\PY{l+s+s1}{\PYZsq{}}\PY{p}{]}\PY{p}{)}\PY{p}{)}\PY{p}{[}\PY{l+m+mi}{37}\PY{p}{]}\PY{p}{,} \PY{n}{color}\PY{o}{=}\PY{l+s+s1}{\PYZsq{}}\PY{l+s+s1}{yellow}\PY{l+s+s1}{\PYZsq{}}\PY{p}{,} \PY{n}{label}\PY{o}{=}\PY{l+s+s1}{\PYZsq{}}\PY{l+s+s1}{high risk period}\PY{l+s+s1}{\PYZsq{}}\PY{p}{)}\PY{p}{;}
         \PY{n}{plt}\PY{o}{.}\PY{n}{scatter}\PY{p}{(}\PY{n+nb}{sorted}\PY{p}{(}\PY{n+nb}{list}\PY{p}{(}\PY{n}{us\PYZus{}vax}\PY{p}{[}\PY{l+s+s1}{\PYZsq{}}\PY{l+s+s1}{date}\PY{l+s+s1}{\PYZsq{}}\PY{p}{]}\PY{p}{)}\PY{p}{)}\PY{p}{[}\PY{l+m+mi}{36}\PY{p}{]}\PY{p}{,} \PY{l+m+mi}{1200000}\PY{p}{,} \PY{n}{s}\PY{o}{=}\PY{l+m+mi}{250}\PY{p}{,} \PY{n}{color}\PY{o}{=}\PY{l+s+s2}{\PYZdq{}}\PY{l+s+s2}{purple}\PY{l+s+s2}{\PYZdq{}}\PY{p}{,} \PY{n}{marker}\PY{o}{=}\PY{l+s+s1}{\PYZsq{}}\PY{l+s+s1}{o}\PY{l+s+s1}{\PYZsq{}}\PY{p}{)}\PY{p}{;}
         \PY{n}{plt}\PY{o}{.}\PY{n}{style}\PY{o}{.}\PY{n}{use}\PY{p}{(}\PY{l+s+s1}{\PYZsq{}}\PY{l+s+s1}{seaborn\PYZhy{}dark}\PY{l+s+s1}{\PYZsq{}}\PY{p}{)}\PY{p}{;} \PY{n}{plt}\PY{o}{.}\PY{n}{legend}\PY{p}{(}\PY{p}{)}\PY{p}{;}
         \PY{n}{plt}\PY{o}{.}\PY{n}{rcParams}\PY{o}{.}\PY{n}{update}\PY{p}{(}\PY{p}{\PYZob{}}\PY{l+s+s1}{\PYZsq{}}\PY{l+s+s1}{font.size}\PY{l+s+s1}{\PYZsq{}}\PY{p}{:} \PY{l+m+mi}{18}\PY{p}{\PYZcb{}}\PY{p}{)}
\end{Verbatim}


    \begin{center}
    \adjustimage{max size={0.9\linewidth}{0.9\paperheight}}{output_53_0.png}
    \end{center}
    { \hspace*{\fill} \\}
    

    % Add a bibliography block to the postdoc
    
    
    
    \end{document}
